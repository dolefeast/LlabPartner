\chapter*{Anexo: Ejemplo para introducir código ISE}

\addcontentsline{toc}{chapter}{Anexo: Ejemplo para introducir código ISE}

\lstdefinelanguage{ISE} {morekeywords={library,use,entity,is,port,in,out,end,
               architecture,of,is,signal,begin,process,then,
               port,downto,if,and,then,else
               },                
sensitive=false,
emph={STD_LOGIC_1164,
STD_LOGIC_ARITH,
STD_LOGIC_UNSIGNED,
IEEE,
ALL,
STD_LOGIC,
STD_LOGIC_VECTOR}, 
emphstyle=\color{magenta},
morecomment=[l]{--},
basicstyle=\small \ttfamily,
   keywordstyle=\color{blue},
   commentstyle=\color{webgreen},
   stringstyle=\color{webtinto},
   numbers=left,
   numberstyle=\tiny\color{gray},
   stepnumber=1,
   numbersep=10pt,
   backgroundcolor=\color{white},
   tabsize=4,
   showspaces=false,
   showstringspaces=false,
%
   frame=LtbR,
     framerule=0.5pt,
     aboveskip=0.5cm,
     framextopmargin=3pt,
     framexbottommargin=3pt,
%     framexleftmargin=1cm,
     framesep=0pt,
     rulesep=.4pt,
     backgroundcolor=\color{gray96},
     rulesepcolor=\color{black}
     }


\renewcommand{\baselinestretch}{1}
 \begin{lstlisting}[language=ISE]
library IEEE;
             use IEEE.STD_LOGIC_1164.ALL;
             use IEEE.STD_LOGIC_ARITH.ALL;
             use IEEE.STD_LOGIC_UNSIGNED.ALL;
-- Uncomment the following library declaration if 
-- instantiating  any Xilinx primitive in this code.
-- library UNISIM;
-- use UNISIM.VComponents.all;
     
entity counter is
	Port ( CLOCK : in  STD_LOGIC;
   	DIRECTION :    in  STD_LOGIC;
  	COUNT_OUT :    out STD_LOGIC_VECTOR (3 downto 0));
end counter;

architecture Behavioral of counter is
signal count_int : std_logic_vector(3 downto 0) := "0000"; 
begin
process (CLOCK)
begin
	if CLOCK='1' and CLOCK'event then
		if DIRECTION='1' then
			count_int <= count_int + 1;
		else
			count_int <= count_int - 1;
		end if;
	end if;
end process;
COUNT_OUT <= count_int;
end Behavioral;
\end{lstlisting}
\renewcommand{\baselinestretch}{1.5}

