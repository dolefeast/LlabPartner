% ----------------------------------------------- 
%  Plantilla LaTeX para el Trabajo Fin de Grado
%         Facultad de Ciencias
%        Universidad de Córdoba
%             Codificación utf8
%------------------------------------------------
%              Abril 2022
%------------------------------------------------
\documentclass[12pt,a4paper]{report}

% Fichero de estilo proporcionado
%%%%%%%%%%%%%%%%%%%%%%%%%%%%%%%%%%%%%%%%%%%%%%%%%%%%
%%
%% Este es el fichero 'estilo_TFG_CienciasUCO_2022.sty'
%%    con el estilo LaTeX para escribir los
%%          TRABAJOS FIN DE GRADO
%%                 de la 
%%          Facultad de Ciencias 
%%         Universidad de Córdoba
%               Abril 2022
%    
%           CODIFICAIÓN utf-8
%%%%%%%%%%%%%%%%%%%%%%%%%%%%%%%%%%%%%%%%%%%%%%%%%%%%%



% Idioma castellano 
\usepackage[utf8]{inputenc}
\usepackage[T1]{fontenc}
\usepackage[spanish,es-tabla]{babel}
%\usepackage{mathpazo} 
%\usepackage{eulervm}


% Tamaño de la página
\usepackage[left=2.5cm,right=2.5cm,top=2cm,bottom=2cm]{geometry}

% Para redefinir algunas cosas del babel en castellano 
\accentedoperators    % Pone acentos a los operadores
\decimalpoint         % Usa el punto como separador decimal
                    

% Paquetes auxiliares  (incluir aquellos que se necesiten)

\usepackage[usenames,dvipsnames]{xcolor} % Para usar colores
\usepackage{tikz}                        % Para dibujar
\usepackage[strict]{changepage}          % Para redefinir márgenes en mitad del documento

% Para matemáticas
\usepackage{amsmath}
\usepackage{amsfonts}
\usepackage{amssymb}

% Para gráficos
\usepackage{graphicx}
\usepackage{subfigure}
\usepackage{float}
%
\usepackage{appendix}   % Contiene mandatos adicionales para apéndices
\usepackage{titlesec}   % Para redefinir capítulos, secciones,...
\usepackage{longtable}  % Para tablas grandes


% Para las cabeceras y pies (luego se redefine cuando haga falta)
\usepackage{fancyhdr}   % Para las cabeceras
\pagestyle{fancy} 
\fancyhf{}
\renewcommand{\headrulewidth}{0pt}
\renewcommand{\footrulewidth}{0pt}

%Para el fondo de la portada
\usepackage{eso-pic}
\newcommand\BackgroundPic{%
\put(0,0){%
\parbox[b][\paperheight]{\paperwidth}{%
\vfill
\centering
\includegraphics[width=\paperwidth,height=\paperheight,%
keepaspectratio]{fondo_portada.png}%
\vfill
}}}

% Estilo de capítulo, secciones...

\titlespacing*{\chapter}{0pt}{50pt}{40pt}
\titleformat{\chapter}[display]
  { \sffamily\bfseries\Large\color{blue}}
  {\filleft\MakeUppercase{\chaptertitlename}\ \Huge\sffamily{\arabic{chapter}}\ 
  }  
  {4ex}
  {\titlerule
   \vspace{2ex}%
   \filright}
  [\vspace{2ex}\hrule\vspace{2pt}%
   \titlerule]
   
   \titleformat{\section}[hang]
{\large \scshape\bfseries}
{\color{blue}  \thesection .}
{1ex}{\color{blue}}[\quad]
{}

  \titleformat{\subsection}[runin]
{ \scshape\bfseries}
{\color{blue}  \thesubsection .}
{1ex}{\color{blue}}[\quad]
{}
   
%%%%%%%%%%%%%%%%%%%%%%%%%%%%%%%%%%%%
% Para los códigos 
\usepackage{listings}

% Para código Matlab
\lstset{language=Matlab,
  keywords={break,case,catch,continue,else,elseif,end,for,function,
      global,if,otherwise,persistent,return,switch,try,while,diary,
      clear,clc,who,whos,help,helpwin,linspace,length,ans,doc,floor,
      ceil,max,min,sum,prod,sort,round,sign,fix,mean,exp,sin,cos,tan,
      input,fprint,load,save,disp,fopen,fclose,inline,function,feval,
      global,return,poly,polyval,roots,solve,fzero, poly2sym,hold,plot,
      polyfit,ppval,spline,interp1,quad,quadl,quadgk,inv,\,lu,cond,spdisgs,
      magic,ode*,ode45,ode23,flops,rref,pinv,chol,zeros,ones,rand,sparse,
      tic,toc,diag,eye,speye,espones,spy,syms,diff,dsolve,simplify,ezplot},
   basicstyle=\small \ttfamily,
   keywordstyle=\color{blue},
   commentstyle=\color{webgreen},
   stringstyle=\color{webtinto},
   numbers=left,
   numberstyle=\tiny\color{gray},
   stepnumber=1,
   numbersep=10pt,
   backgroundcolor=\color{white},
   tabsize=4,
   showspaces=false,
   showstringspaces=false,
%
   frame=LtbR,
     framerule=0.5pt,
     aboveskip=0.5cm,
     framextopmargin=3pt,
     framexbottommargin=3pt,
%     framexleftmargin=1cm,
     framesep=0pt,
     rulesep=.4pt,
     backgroundcolor=\color{gray96},
     rulesepcolor=\color{black}}
     
     
 % Para código ISE
     
\lstdefinelanguage{ISE} {morekeywords={library,use,entity,is,port,in,out,end,
               architecture,of,is,signal,begin,process,then,
               port,downto,if,and,then,else
               },                
sensitive=false,
emph={STD_LOGIC_1164,
STD_LOGIC_ARITH,
STD_LOGIC_UNSIGNED,
IEEE,
ALL,
STD_LOGIC,
STD_LOGIC_VECTOR}, 
emphstyle=\color{magenta},
morecomment=[l]{--},
basicstyle=\small \ttfamily,
   keywordstyle=\color{blue},
   commentstyle=\color{webgreen},
   stringstyle=\color{webtinto},
   numbers=left,
   numberstyle=\tiny\color{gray},
   stepnumber=1,
   numbersep=10pt,
   backgroundcolor=\color{white},
   tabsize=4,
   showspaces=false,
   showstringspaces=false,
%
   frame=LtbR,
     framerule=0.5pt,
     aboveskip=0.5cm,
     framextopmargin=3pt,
     framexbottommargin=3pt,
%     framexleftmargin=1cm,
     framesep=0pt,
     rulesep=.4pt,
     backgroundcolor=\color{gray96},
     rulesepcolor=\color{black}
     }
     
%para generar índice con hipervínculos
\usepackage{hyperref} 



\hypersetup{
    colorlinks,
    citecolor=webgreen,
    filecolor=black,
    linkcolor=webblue,
    urlcolor=webtinto,
}


% Colores definidos
\definecolor{azul}{rgb}{65,105,225}
\definecolor{webgreen}{rgb}{0,.5,0}
\definecolor{webgray}{rgb}{.753,.753,.753}
\definecolor{webblue}{rgb}{0,0,.8}
\definecolor{webtinto}{rgb}{0.73,0.00,0.00}
\definecolor{gray96}{gray}{.96}

% Algunas redefiniciones 
%\renewcommand{\contentsname}{Índice general}
%\renewcommand{\partname}{Parte}
%\renewcommand{\appendixname}{Apéndice}
%\renewcommand{\figurename}{Figura}
%\renewcommand{\listfigurename}{Índice de figuras}
%\renewcommand{\chaptername}{Capítulo}
%\renewcommand{\bibname}{Bibliografía}




%%%% Nuevas definiciones %%%%%%%%%%%%%%%%%%%%%%%%%%%%%%%

% Ejemplos de algunas definiciones 
\def \dint{\displaystyle \int}
\def \gt {\gamma(t)}        %  gamma(t)
\def \st {\tilde{\sigma}}   %  sigma tilde
\def \sg {\sigma_{\gamma}}  %  sigma sub gamma
\def \omt {\omega(t)}       %  omega(t)
\def \bu {\mathbf{u}}       %  u negrilla
\def \R {\mathbb{R}}        %  números reales


%--------------------------------------------------------------------------------------
% CAMBIAR LOS DATOS EN CADA CASO 

\newcommand{\titulacion}{Grado de Física}                  % Grado de Biología
                                                      % Grado de Bioquímica
                                                      % Grado de Ciencias Ambientales
                                                      % Grado de Física
                                                      % Grado de Química
                                                      
\newcommand{\titulo}{Oscilaciones Bariónicas Acústicas en Universos con Curvatura} % Título
\newcommand{\codigo}{FS22-17-FSC}      % Código
\newcommand{\tipo}{Trabajo teórico-práctico general}             % Trabajo teórico--práctico general
                                          		     % Trabajo de iniciación a la investigación
                                                      % Trabajo en empresa
                                                      % Idea de negocio
                                                      % Propuesta científico--técnica
                                                      % Trabajo docente
                                          
\newcommand{\autor}{Santiago Sanz Wuhl}     % Autor
\newcommand{\fecha}{Fecha de entrega}                 % Fecha de entrega
%--------------------------------------------------------------------------------------
\newcommand{\nomfig}{Figura}                         
\newcommand{\nomfigs}{figuras} 

% Si se quiere que las figuras se denominen ilustraciones comentar las dos órdenes
% anteriores y descomentar las dos que siguen 

%\newcommand{\nomfig}{Ilustración}                         
%\newcommand{\nomfigs}{ilustraciones}   


%--------------------------------------------------------------------------------------

% Empieza el documento
\begin{document}

% No cambiar  %%%%%%%%%%%%%%%%%%%%%%%%%%%%%%%%%%%%%%%%%%%%%%%%%%%%%%%%%%%%
%%%%%%%%%%    NO CAMBIAR   %%%%%%%%%%%%%%%%%%%%%%%%%%%%%
% Inclusión automática de la portada, tribunal e índices
%%%%%%%%%%    NO CAMBIAR   %%%%%%%%%%%%%%%%%%%%%%%%%%%%%%

% PORTADA----------------------------------
\AddToShipoutPicture*{\BackgroundPic}
\begin{titlepage}
\begin{center}
\Large UNIVERSIDAD DE CÓRDOBA\\[0.5 cm]
\large  \textbf{Facultad de Ciencias}\\[1.25 cm]
\large \textbf{\titulacion}\\[1.25 cm]
\Large  Trabajo Fin de Grado\\[2.25 cm]
\Huge   \titulo
\end{center}
\vspace{1.25cm}
\noindent {\large Código del TFG: \textbf{\color{blue}\codigo}}\\[0.3cm]
\noindent {\large Tipo de TFG: \textbf{\color{blue}\tipo}}\\[0.5cm]
{\color{blue}\hrule}
\vspace{0.5cm}
\noindent \Large{Autor: \autor}
\vfill
\begin{center}
 \includegraphics[width=6cm]{logo_ciencias.png} 
\end{center}
\vfill
\rightline{\fecha}
\end{titlepage}
\renewcommand{\baselinestretch}{0.9}
\ClearShipoutPicture

 %%%%%%%%%%%%%%%%%%%%%%%%%%%%%%%%%%%%%%%%%%%%%%%%%%%%%%%%%%%%%%%%%%




   % Dejar esta línea (incluirá el fichero 		 
%                                  portada.tex
% 	                              proporcionado, que no hay que modificar
% Fin de No cambiar %%%%%%%%%%%%%%%%%%%%%%%%%%%%%%%%%%%%%%%%%%%%%%%%%%%%%%%


\fancypagestyle{plain}{
\fancyfoot[R]{\large{pág. \thepage}}
}

\pagestyle{fancy} 

\chapter*{Agradecimientos}

Incluir los agradecimientos, si procede. % Agradecimientos y/o dedicatoria si procede (si no comentar esta línea)

% No cambiar  %%%%%%%%%%%%%%%%%%%%%%%%%%%%%%%%%%%%%%%%%%%%%%%%%%%%%%%%%%%%
% INDICES-----------------------------------

% Genera el índice de contenidos------------

\fancypagestyle{plain}{
\fancyfoot[R]{\small pág. \thepage} 
}
\tableofcontents
\addcontentsline{toc}{chapter}{Índice general}

\renewcommand{\baselinestretch}{1.5}

% Genera el índice de figuras---------------
\fancypagestyle{plain}{
\fancyfoot[R]{\small pág. \thepage} 
}
\renewcommand{\listfigurename}{Índice de \nomfigs}
\renewcommand{\figurename}{\nomfig}

\listoffigures
\addcontentsline{toc}{chapter}{Índice de \nomfigs}


% Genera el índice de tablas---------------
\fancypagestyle{plain}{
\fancyfoot[R]{\small pág. \thepage}
}
\listoftables
\addcontentsline{toc}{chapter}{Índice de tablas}


% Para las cabeceras del resto

\pagestyle{fancy} 
\fancyhf{}
\fancyfoot[R]{\small pág. \thepage}
\renewcommand{\headrulewidth}{0pt}
\renewcommand{\footrulewidth}{0pt}  % Dejar esta línea (incluirá el fichero 		 
%                                  indices.tex
% 	                              proporcionado, que no hay que modificar
% Fin de No cambiar %%%%%%%%%%%%%%%%%%%%%%%%%%%%%%%%%%%%%%%%%%%%%%%%%%%%%%%


\chapter*{Resumen}
\addcontentsline{toc}{chapter}{Resumen. Palabras clave}

Escriba aquí un resumen de la memoria en castellano que contenga entre 100 y 300 palabras. Las palabras clave serán entre 3 y 6.

\paragraph{Palabras clave:} palabra clave 1; palabra clave 2; palabra clave 3; palabra clave 4 









\chapter*{Abstract}
\addcontentsline{toc}{chapter}{Abstract. Keywords}

Insert here the abstract of the report with an extension between 100 and 300 words. 

\paragraph{Keywords:} keyword1; keyword2; keyword3; keyword4
          % Resumen (insertar en el fichero resumen.tex proporcionado,
%                                     el resumen en español e inglés)

% empiezan los capítulos
\chapter{Introducción}


\section{Fundamento teórico}
\subsection{Oscilaciones Bariónicas Acústicas}
En los primeros cientos de miles de años del universo, éste existía en un estado hiper denso conocido como el plasma primordial, formado por (hasta lo que sabemos) materia oscura, bariones y fotones.
Debido a la alta concentración de materia del universo temprano, había una fortísima atracción gravitatoria, contrarrestada por la presión de radiación debida al Efecto Thomson. 

Estas fuerzas generaban colisiones entre las partículas del plasma primordial que se propagaban en forma de ondas esféricas, con un mecanismo casi idéntico al del sonido en el aire. 

Debido a la expansión del universo, pasado cierto tiempo la densidad del mismo no será lo suficientemente alta como para continuar permitiendo la propagación de estas ondas esféricas, congelándolas así en el tiempo. 
Al ser estas ondas picos de densidad de materia, será en estas zonas donde haya la mayor probabilidad de formación de galaxias. 

Al conocer el tamaño 'comoving'\footnote{'Comoving' hace referencia a lo que mediríamos si el universo no se hubiese expandido} (aproximadamente 150Mpc)\cite[Eisenstein2004] de estas ondas esféricas congeladas en el tiempo, si conseguimos medir su tamaño actual 'físico' (es decir, considerando la expansión del universo') podremos usar esos resultados para tomar medidas más y más precisas de objetos muy distantes.

\subsection{Efecto de la curvatura del universo}

A día de hoy, todos los estudios BAO que se han realizado han asumido un universo sin curvatura. Esto es, aunque el universo presenta curvatura de manera local debido a las concentraciones de masa, el universo es plano de manera asintótica.





\subsection{Estudio de las curvas BAO}

       % Añadir los diferentes capítulos con el mismo formato que tiene
%                           el fichero capitulo1.tex)
%\include{capitulo2} 
% .......
\chapter{Resultados}

\section{Cálculo de observables}

\begin{figure}[b]
	\centering
	\includegraphics[width=0.7\textwidth]{../figs/DA_DH_flat.pdf}
	\caption{Cálculo de los observables cosmológicos para diferentes cosmologías}
	\label{fig:DA_DH_flat}
\end{figure}


\chapter*{Conclusiones}
\addcontentsline{toc}{chapter}{Conclusiones}

En este trabajo ...


\chapter*{Conclusions}
\addcontentsline{toc}{chapter}{Conclusions}

In this work ...       % Conclusiones (insertar en el fichero conclusion.tex proporcionado,
%                                          las conclusiones en español e inglés)
 
%%%%%%%%%%%%%%%%%%%%%%%%%%%%%%%%%%%%%%%%%%%%%%%%%%%%%%%%%%
%%%%  LA BIBLIOGRAFÍA %%%%%%%%%%%%%%%%%%%%%%%
%%%%%%%%%%%%%%%%%%%%%%%%%%%%%%%%%%%%%%%%%%%%%%%%%%%%%%%%%

\addcontentsline{toc}{chapter}{Bibliografía}
\begin{thebibliography}{999}


\bibitem{Bellomo2000} Bellomo,~N., Preziosi, L. , Modelling and mathematical problems
related to tumor evolution and its interaction with the immune
system. \textit{Mathematical and Computer Modelling}, \textbf{32},  pp. 413--452, 2000.

\bibitem{Arfken2005} Arfken,~G.B., Weber,~H.J., Harris,~F.E. \textit{Mathematical Methods for Physicists}, Sixth Ed.: A Comprehensive Guide, Academic Press, 2005.


\end{thebibliography}     % Bibliografía (insertar en el fichero bibliografia.tex proporcionado, 
%                                          la bibliografia con bibitem)

%----------------------------------------------------------
% Si se tiene una base de datos bibliográfica
% estilo de bibliografía: plana, alfa...
%\nocite{*}
%\bibliographystyle{plain}
%
%% Bibliografía---------------------------
%% Añade la bibliografía al índice
%
%%se incluye la bibliografía. Archivo de tipo .bib (bibtex)
%\bibliography{bib/miaumiau}
%----------------------------------------------------------

% Apéndices (si los hubiera)
\appendix
\chapter*{Anexo: Ejemplo para introducir código Matlab}
\addcontentsline{toc}{chapter}{Anexo: Ejemplo para introducir código Matlab}

\renewcommand{\baselinestretch}{1}
\begin{lstlisting}[language=Matlab]
%% 3-D Plots
% Three-dimensional plots typically display a surface 
% defined by a function in two variables, z = f(x,y) .
%%
% To evaluate z, first create a set of (x,y) points 
% over the domain of the function using meshgrid.
	[X,Y] = meshgrid(-2:.2:2);                                
	Z = X .* exp(-X.^2 - Y.^2);
%%
% Then, create a surface plot.
	surf(X,Y,Z)
%%
% Both the surf function and its companion mesh display 
% surfaces in three dimensions. surf displays both the 
% connecting lines and the faces of the surface in color. 
% Mesh produces wireframe surfaces that color only the 
%lines connecting the defining points.

\end{lstlisting}
\renewcommand{\baselinestretch}{1.5}



    % Añadir los diferentes anexos, si los hubiera, con el mismo formato que tiene
%                     el fichero anexo1.tex)
\chapter*{Anexo: Ejemplo para introducir código ISE}

\addcontentsline{toc}{chapter}{Anexo: Ejemplo para introducir código ISE}

\lstdefinelanguage{ISE} {morekeywords={library,use,entity,is,port,in,out,end,
               architecture,of,is,signal,begin,process,then,
               port,downto,if,and,then,else
               },                
sensitive=false,
emph={STD_LOGIC_1164,
STD_LOGIC_ARITH,
STD_LOGIC_UNSIGNED,
IEEE,
ALL,
STD_LOGIC,
STD_LOGIC_VECTOR}, 
emphstyle=\color{magenta},
morecomment=[l]{--},
basicstyle=\small \ttfamily,
   keywordstyle=\color{blue},
   commentstyle=\color{webgreen},
   stringstyle=\color{webtinto},
   numbers=left,
   numberstyle=\tiny\color{gray},
   stepnumber=1,
   numbersep=10pt,
   backgroundcolor=\color{white},
   tabsize=4,
   showspaces=false,
   showstringspaces=false,
%
   frame=LtbR,
     framerule=0.5pt,
     aboveskip=0.5cm,
     framextopmargin=3pt,
     framexbottommargin=3pt,
%     framexleftmargin=1cm,
     framesep=0pt,
     rulesep=.4pt,
     backgroundcolor=\color{gray96},
     rulesepcolor=\color{black}
     }


\renewcommand{\baselinestretch}{1}
 \begin{lstlisting}[language=ISE]
library IEEE;
             use IEEE.STD_LOGIC_1164.ALL;
             use IEEE.STD_LOGIC_ARITH.ALL;
             use IEEE.STD_LOGIC_UNSIGNED.ALL;
-- Uncomment the following library declaration if 
-- instantiating  any Xilinx primitive in this code.
-- library UNISIM;
-- use UNISIM.VComponents.all;
     
entity counter is
	Port ( CLOCK : in  STD_LOGIC;
   	DIRECTION :    in  STD_LOGIC;
  	COUNT_OUT :    out STD_LOGIC_VECTOR (3 downto 0));
end counter;

architecture Behavioral of counter is
signal count_int : std_logic_vector(3 downto 0) := "0000"; 
begin
process (CLOCK)
begin
	if CLOCK='1' and CLOCK'event then
		if DIRECTION='1' then
			count_int <= count_int + 1;
		else
			count_int <= count_int - 1;
		end if;
	end if;
end process;
COUNT_OUT <= count_int;
end Behavioral;
\end{lstlisting}
\renewcommand{\baselinestretch}{1.5}



% Termina el documento
\end{document}
