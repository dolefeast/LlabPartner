\chapter{Introducción}


\section{Fundamento teórico}
\subsection{Oscilaciones Bariónicas Acústicas}
En los primeros cientos de miles de años del universo, éste existía en un estado hiper denso conocido como el plasma primordial, formado por (hasta lo que sabemos) materia oscura, bariones y fotones.
Debido a la alta concentración de materia del universo temprano, había una fortísima atracción gravitatoria, contrarrestada por la presión de radiación debida al Efecto Thomson. 

Estas fuerzas generaban colisiones entre las partículas del plasma primordial que se propagaban en forma de ondas esféricas, con un mecanismo casi idéntico al del sonido en el aire. 

Debido a la expansión del universo, pasado cierto tiempo la densidad del mismo no será lo suficientemente alta como para continuar permitiendo la propagación de estas ondas esféricas, congelándolas así en el tiempo. 
Al ser estas ondas picos de densidad de materia, será en estas zonas donde haya la mayor probabilidad de formación de galaxias. 

Al conocer el tamaño 'comoving'\footnote{'Comoving' hace referencia a lo que mediríamos si el universo no se hubiese expandido} (aproximadamente 150Mpc)\cite[Eisenstein2004] de estas ondas esféricas congeladas en el tiempo, si conseguimos medir su tamaño actual 'físico' (es decir, considerando la expansión del universo') podremos usar esos resultados para tomar medidas más y más precisas de objetos muy distantes.

\subsection{Efecto de la curvatura del universo}

A día de hoy, todos los estudios BAO que se han realizado han asumido un universo sin curvatura. Esto es, aunque el universo presenta curvatura de manera local debido a las concentraciones de masa, el universo es plano de manera asintótica.





\subsection{Estudio de las curvas BAO}

